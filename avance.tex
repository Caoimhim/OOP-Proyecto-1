\documentclass[12pt]{article} 
\usepackage{graphicx}
\usepackage[utf8]{inputenc}

\author{Ing. Yolanda Martínez Treviño \\ Programación Orientada a Objetos}
\title{\textbf{Avance de Proyecto 1} \\Kevin Chinchilla A00825945}
\date{08 de febrero del 2019}

\begin{document}

\begin{figure}[t]
\includegraphics[height=1in,keepaspectratio]{../../Tec.png}
\centering
\end{figure}
\maketitle

\pagebreak

\section{Tags de HTLML}

\begin{itemize}
	\item \textless !DOCTYPE html\textgreater \\
		Indica que el tipo de documente es html.
	\item \textless html\textgreater y \textless /html\textgreater  \\
		Indica el inicio del documento html y el final del documento html.
	\item \textless head\textgreater  y \textless /head\textgreater  \\
		Entre ellas se contiene la información y las tags que brindan información acerca del documento.

	\item \textless title\textgreater  y \textless /title\textgreater \\
		Entre ellas se escribe el título del documento.

	\item \textless body\textgreater  y \textless /body\textgreater \\
		Entre ellas se escribe el contenido del documento.

	\item \textless h1 … h6\textgreater  y \textless /h1 … /h6\textgreater  \\
		Entre ellas se escriben subtítulos o headers de diferentes niveles, siendo h1 el más alto.

	\item \textless br\textgreater  \\
		Se utiliza para saltarse una línea, cuándo hay información que se requiere colocar en más de una línea.

	\item \textless hr\textgreater \\
		Se utiliza para crear un \textit{break} entre secciones. En la mayoría de los browsers, esto genera una línea separando las secciones.

	\item \textless strong\textgreater  y \textless /strong\textgreater  \\
		Se utilizan para poner énfasis fuerte, lo que resulta en texto en negrillas, en la mayoría de los browsers.

\end{itemize}

\end{document}
